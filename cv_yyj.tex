%%%%%%%%%%%%%%%%%%%%%%%%%%%%%%%%%%%%%%%%%
% Medium Length Professional CV
% LaTeX Template
% Version 2.0 (8/5/13)
%
% This template has been downloaded from:
% http://www.LaTeXTemplates.com
%
% Original author:
% Trey Hunner (http://www.treyhunner.com/)
%
% Important note:
% This template requires the resume.cls file to be in the same directory as the
% .tex file. The resume.cls file provides the resume style used for structuring the
% document.
%
%%%%%%%%%%%%%%%%%%%%%%%%%%%%%%%%%%%%%%%%%

%----------------------------------------------------------------------------------------
%	PACKAGES AND OTHER DOCUMENT CONFIGURATIONS
%----------------------------------------------------------------------------------------

\documentclass{resume} % Use the custom resume.cls style

\usepackage[left=0.75in,top=0.5in,right=0.75in,bottom=0.5in]{geometry} % Document margins
\usepackage{hyperref}
\usepackage{fancyhdr}
\renewcommand{\headrulewidth}{0pt}
\fancyhead{}
% \fancyfoot[C]{Page \thepage}
\pagestyle{fancy}


%\newcommand{\tab}[1]{\hspace{.2667\textwidth}\rlap{#1}}
\name{Yanjun Yang} % Your name\newcommand{\itab}[1]{\hspace{0em}\rlap{#1}}

\address{yanjun.yang@ed.ac.uk} % Your email address
\address{blog.yanaerobe.top}
\address{+44 7960 011793/+86 150 7499 5687} % Your phone number

\begin{document}

%----------------------------------------------------------------------------------------
%	EDUCATION SECTION
%----------------------------------------------------------------------------------------

\begin{rSection}{Education}

	\begin{rSubsection}{The University of Edinburgh}{Sept, 2022 - present}{PhD Candidate, IMNS}{Edinburgh, Scotland}
		\item Supervisors: Alex Serb, Themis Prodromakis
		\item Research theme: Cognitive processing system implementation based on memristor associative memory
	\end{rSubsection}
	
	\begin{rSubsection}{Nanyang Technological University}{Aug, 2021 - Aug, 2022}{MSc (Electronics)}{Singapore}
		\item Core units taken: Digital Integrated Circuit Design, Electromagnetic Compability Design, Genetic Algorithms and Machine Learning, Advanced Wafer Processing, Integrated Circuit Packaging
	\end{rSubsection}

	\begin{rSubsection}{Tongji University}{Sept, 2017 - July, 2021}{BEng. in Electronic Science and Technology}{Shanghai, China}
		\item Core units taken: Design and Analysis of Digital Integrated Circuits, Computer Architecture, Principles for Design of Integrated Circuit Chips, Embedded Systems
	\end{rSubsection}

\end{rSection}

\begin{rSection}{Current Projects}
	
	\begin{rSubsection}{ASOCA3-FPGA}{Jan, 2024 - present}{The University of Edinburgh}{PhD Sub-project}
		\item Developing a graph database accelerator system at million-entry level
		\item Fully in charge of architecture iterating
	\end{rSubsection}
	
	\begin{rSubsection}{ASOCat}{Sept, 2022 - present}{The University of Edinburgh}{PhD Sub-project}
		\item Building a Copycat-based cognitive model compatible with an associative memory chip
		\item Designing and implementing its software/hardware interface
	\end{rSubsection}

\end{rSection}

\begin{rSection}{Publications}

	% \begin{rSubsection}{Views: A Hardware-friendly Graph Database Architecture For Storing Semantic Information}{Apr, 2023 - Dec, 2023}{The University of Edinburgh}{PhD Sub-project}
	% 	\item Proof and theoretical simulation of novel graph database structures
	% 	\item Demonstration semantic reasoning and cognitive modeling under the proposed architecture
	% \end{rSubsection}
	
	\begin{rSubsection}{A Resource-efficient Dually-addressable Memory Architecture\\on FPGA}{Apr, 2023 - Dec, 2023}{The University of Edinburgh}{ISCAS 2024 (\textit{under review})}
		\item Presented a resource-efficient BRAM-based DAM architecture on FPGA
		\item Implemented onto Virtex-7/Virtex UltraScale+ FPGAs with 100\% storage efficiency
	\end{rSubsection}
	
	\begin{rSubsection}{A Single-layer Wideband Microwave Absorber with Reactive Screen, and\\A Novel Design of Microwave Absorber for Reduction of Radar\\Cross Section}{July, 2020}{Tongji University}{AP-S/URSI 2020}
		\item Carried out antenna simulation under HFSS and experimental data analyses
		% \item Published two academic papers accepted by IEEE AP-S/URSI 2020 as the second author
	\end{rSubsection}

\end{rSection}

\newpage

\begin{rSection}{Research Experience}
	
	\begin{rSubsection}{Saliency detector}{Feb, 2024 - July, 2024}{The University of Edinburgh, STMicro}{PhD Sub-project}
		\item Developed a hybrid defect detection and classfication prototype
		\item Produced a result analysis and development report
	\end{rSubsection}
	
	\begin{rSubsection}{ASOCA2}{Nov, 2022 - Dec, 2023}{The University of Edinburgh}{PhD Sub-project}
		\item Successful tape-out of a memristor-based associative memory SoC
		\item Core digital hardware developer and verification engineer
	\end{rSubsection}
	
	\begin{rSubsection}{ReMap: a Mitchell-based logarithmic conversion circuit}{Aug, 2021 - July, 2022}{Nanyang Technological University}{MSc Dissertation Project}
		\item Optimised a Mitchell-based binary logarithmic approximation method
		\item Implemented and evaluated corresponding integrated circuits
	\end{rSubsection}

	\begin{rSubsection}{Design of a hierarchical memory management mechanism}{Aug, 2021 - Jan, 2022}{Tongji University}{Part-time Internship}
		\item Led a hierarchical SRAM-flash interface design with page replacement algorithm
		\item Applied the design on an automobile-orientated MCU
	\end{rSubsection}
	
	\begin{rSubsection}{CoNM: Core of Normal Microarchitecture}{Mar, 2021 - Jun, 2021}{Tongji University}{Graduation Design}
		\item Designed a four-stage RV32I CPU core with static branch prediction in Verilog
		\item Implemented onto PYNQ-Z1 FPGA board for a successful verification
	\end{rSubsection}

\end{rSection}

\begin{rSection}{Professional Skills}

	Familiar with both FPGA and digital ASIC workflows. 
	
	\begin{rSubsection}{Programming Languages}{}{}{}
		\item Proficient in SystemVerilog/Verilog
		\item Skilled in Python, Bash, Tcl
		\item Competent in C/C++, VHDL
		\item Developing skills in CHISEL
	\end{rSubsection}

	\begin{rSubsection}{Professional Software}{}{}{}
		\item Skilled in Synopsys, Vivado, Cadence
		\item Good command of MATLAB, ModelSim
		\item Good knowledge of HFSS, ISE, Keil
	\end{rSubsection}

	\begin{rSubsection}{Languages}{}{Native in Mandarin and New Xiang, proficient in English}{}
		\item IELTS 8.0/9.0 (2020), equivalent to CEFR level C1
		\item Elementary reading proficiency of French
	\end{rSubsection}

\end{rSection}

\begin{rSection}{Workshops}
	
	\begin{rSubsection}{2nd International Workshop on Deep Learning meets Neuromorphic Hardware}{September, 2024}{Vilnius, Lituania}
		\item A modular graph database accelerator
	\end{rSubsection}

\end{rSection}

\begin{rSection}{Teaching}
	
	\begin{rSubsection}{The University of Edinburgh}{1st Semester, 2024}{}{}
		\item Demonstrator - Digital Systems Design 2
		\item Marker - Embedded Systems Design 2
	\end{rSubsection}
	
	\begin{rSubsection}{Digital Systems Design 3}{2nd Semester, 2024}{}{}
		\item Demonstrator - Digital Systems Design 3
		% \item Marker - Microelectronics
	\end{rSubsection}
	
\end{rSection}

\begin{rSection}{Honours and Awards}
	
	\begin{rSubsection}{Tongji University}{1st Semester, 2017}{Advanced Summer Internship Individual}{}
		\item Awarded for the great performance during summer internship
	\end{rSubsection}
		
	\begin{rSubsection}{Tongji University}{1st Semester, 2019}{Outstanding Student Cadre}{}
		\item Awarded for the outstanding work as minister of Rights and Welfare Department in Students' Union
	\end{rSubsection}

\end{rSection}

\begin{rSection}{Interests}
	\item Prefer \textit{Scotches}.
	\item $\frac{1}{2}$ geek, $\frac{1}{2}$ bartender.
	% \item Neovim user under WSL2.
	\item Jay Chou and Aska Yang fan. 
	\item (Was) Practising the harmonica.
	\item Reads Kafka, Hai Zi, Shelley and Allan Poe.
	\item Always plannig to do something in film criticism and lyric writing.
\end{rSection}

\end{document}