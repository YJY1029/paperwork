%%%%%%%%%%%%%%%%%%%%%%%%%%%%%%%%%%%%%%%%%
% Medium Length Professional CV
% LaTeX Template
% Version 2.0 (8/5/13)
%
% This template has been downloaded from:
% http://www.LaTeXTemplates.com
%
% Original author:
% Trey Hunner (http://www.treyhunner.com/)
%
% Important note:
% This template requires the resume.cls file to be in the same directory as the
% .tex file. The resume.cls file provides the resume style used for structuring the
% document.
%
%%%%%%%%%%%%%%%%%%%%%%%%%%%%%%%%%%%%%%%%%

%----------------------------------------------------------------------------------------
%	PACKAGES AND OTHER DOCUMENT CONFIGURATIONS
%----------------------------------------------------------------------------------------

\documentclass{resume} % Use the custom resume.cls style

\usepackage[left=0.75in,top=0.5in,right=0.75in,bottom=0.5in]{geometry} % Document margins
\usepackage{hyperref}
\usepackage{fancyhdr}
\renewcommand{\headrulewidth}{0pt}
\fancyhead{}
% \fancyfoot[C]{Page \thepage}
\pagestyle{fancy}


%\newcommand{\tab}[1]{\hspace{.2667\textwidth}\rlap{#1}}
\name{Yanjun Yang} % Your name\newcommand{\itab}[1]{\hspace{0em}\rlap{#1}}

\address{yanjun.yang@ed.ac.uk} % Your email address
\address{yanjyang.github.io}
\address{+44 7960 011793/+86 150 7499 5687} % Your phone number

\begin{document}

%----------------------------------------------------------------------------------------
%	EDUCATION SECTION
%----------------------------------------------------------------------------------------

\begin{rSection}{Education}
	
	\begin{rSubsection}{Tongji University}{Sept, 2017 - July, 2021}{BEng. in Electronic Science and Technology}{Shanghai, China}
		\item Core units taken: Design and Analysis of Digital Integrated Circuits, Computer Architecture, Principles for Design of Integrated Circuit Chips, Embedded Systems
	\end{rSubsection}

	\begin{rSubsection}{Nanyang Technological University}{Aug, 2021 - Aug, 2022}{MSc (Electronics)}{Singapore}
		\item Core units taken: Digital Integrated Circuit Design, Electromagnetic Compability Design, Genetic Algorithms and Machine Learning, Advanced Wafer Processing, Integrated Circuit Packaging
	\end{rSubsection}

	\begin{rSubsection}{The University of Edinburgh}{Sept, 2022 - present}{PhD Student, IMNS}{Edinburgh, Scotland}
		\item Supervisors: Alex Serb, Themis Prodromakis
		\item Research theme: High-level design of a symbolic AI system based on Douglas Hofstadter
	\end{rSubsection}

\end{rSection}

\begin{rSection}{Research Experience}
	
	\begin{rSubsection}{ReMap: a Mitchell-based logarithmic conversion circuit}{Aug, 2021 - July, 2022}{Nanyang Technological University}{MSc Dissertation Project}
		\item Researched on algorithm optimizations of a Mitchell-based binary logarithmic approximation method
		\item Implemented and evaluated corresponding integrated circuits
	\end{rSubsection}

	\begin{rSubsection}{Design of a hierarchical memory management mechanism}{Aug, 2021 - Jan, 2022}{Tongji University}{Part-time Internship}
		\item In charge of a hierarchical SRAM-flash interface design with page replacement algorithm
		\item Applying the design on an automobile-orientated Cortex-M3 MCU
	\end{rSubsection}
	
	\begin{rSubsection}{CoNM: Core of Normal Microarchitecture}{Mar, 2021 - Jun, 2021}{Tongji University}{Graduation Design}
		\item Designed a four-stage 50MHz RV32I CPU core with a static branch prediction in Verilog
		\item Transplanted the core onto PYNQ-Z1 FPGA for a successful verification
	\end{rSubsection}

	\begin{rSubsection}{A Single-layer Wideband Microwave Absorber with Reactive Screen, A Novel Design of Microwave Absorber for Reduction of Radar Cross Section}{Dec, 2019 - May, 2020}{Tongji University}{Second Author}
		\item In charge of HFSS antenna simulation and analyses of experimental data
		\item Published two academic papers accepted by IEEE AP-S/URSI 2020 as the second author
	\end{rSubsection}

\end{rSection}

\newpage

\begin{rSection}{Current Project}
	
	\begin{rSubsection}{EasyCopy}{Sept, 2022 - present}{The University of Edinburgh}{PhD Sub-project}
		\item Building a FORTE-compatible version of D. Hofstadter's Copycat cognitive model
		\item Designing and implementing its software/hardware interface
	\end{rSubsection}

\end{rSection}

\begin{rSection}{Honours and Awards}
	
	\begin{rSubsection}{Tongji University}{1st Semester, 2017}{Advanced Summer Internship Individual}{}
		\item Awarded for the great performance during summer internship
	\end{rSubsection}
		
	\begin{rSubsection}{Tongji University}{1st Semester, 2019}{Outstanding Student Cadre}{}
		\item Awarded for the outstanding work as minister of the Rights and Welfare Department in Students' Union
	\end{rSubsection}

\end{rSection}

\begin{rSection}{Professional Skills}

	Familiar with both ASIC and FPGA design workflows. 
	
	\begin{rSubsection}{Programming Languages}{}{}{}
		\item Proficient in Verilog
		\item Competent in C++, C, VHDL
		\item Developing skills in CHISEL, Python
	\end{rSubsection}

	\begin{rSubsection}{Professional Software}{}{}{}
		\item Skilled in Synopsys, ModelSim, MATLAB 
		\item Good command of Vivado, Cadence
		\item Good knowledge of HFSS, ISE, Keil
	\end{rSubsection}

	\begin{rSubsection}{Languages}{}{Native in Chinese, proficient in English}{}
		\item IELTS 8.0/9.0, equivalent to CEFR level C1
		\item Elementary reading proficiency of French
	\end{rSubsection}

\end{rSection}

\begin{rSection}{Interests}
	\item Had thought of becoming a writer and polyglot. 
	\item A Jay Chou and Aska Yang fan. 
	\item Very interested but rather ignorant in classical music, operas and cocktails. 
\end{rSection}

\end{document}
