%%%%%%%%%%%%%%%%%%%%%%%%%%%%%%%%%%%%%%%%%
% Medium Length Professional CV
% LaTeX Template
% Version 2.0 (8/5/13)
%
% This template has been downloaded from:
% http://www.LaTeXTemplates.com
%
% Original author:
% Trey Hunner (http://www.treyhunner.com/)
%
% Important note:
% This template requires the resume.cls file to be in the same directory as the
% .tex file. The resume.cls file provides the resume style used for structuring the
% document.
%
%%%%%%%%%%%%%%%%%%%%%%%%%%%%%%%%%%%%%%%%%

%----------------------------------------------------------------------------------------
%	PACKAGES AND OTHER DOCUMENT CONFIGURATIONS
%----------------------------------------------------------------------------------------

\documentclass{resume} % Use the custom resume.cls style

\usepackage[left=0.75in,top=0.5in,right=0.75in,bottom=0.5in]{geometry} % Document margins
\usepackage{hyperref}
\usepackage{fancyhdr}
\usepackage[UTF8]{ctex}
\renewcommand{\headrulewidth}{0pt}
\fancyhead{}
% \fancyfoot[C]{Page \thepage}
\pagestyle{fancy}


%\newcommand{\tab}[1]{\hspace{.2667\textwidth}\rlap{#1}}
\name{杨彦隽} % Your name\newcommand{\itab}[1]{\hspace{0em}\rlap{#1}}

\address{https://github.com/YJY1029}
\address{+65 9120 3467 / +86 15074995687} % Your phone number
\address{yyang039@e.ntu.edu.sg} % Your email address

\begin{document}

%----------------------------------------------------------------------------------------
%	EDUCATION SECTION
%----------------------------------------------------------------------------------------

\begin{rSection}{教育经历}
	
	\begin{rSubsection}{同济大学}{2017年9月 - 2021年7月}{电子科学与技术,学士}{中国上海}
		\item 平均绩点:84.8\% 
		\item 主要课程:数字集成电路分析与设计,计算机体系结构,模拟集成电路分析与设计,集成系统芯片设计原理,半导体器件原理,嵌入式系统
	\end{rSubsection}

	\begin{rSubsection}{南洋理工大学}{2021年8月 - }{Msc (Electronics)}{新加坡}
		\item 平均绩点: 4.63/5.00
		\item 主要课程: Digital Integrated Circuits Design, Advanced Topics in Semiconductor Devices, Electromagnetic Compability Design, LED Lighting \& Display Technologies
	\end{rSubsection}

\end{rSection}

\begin{rSection}{研究经历}

	\begin{rSubsection}{分级存储管理机制设计}{2021年8月- 2022年1月}{同济大学}{实习研究助理}
		\item 主持设计一种使用页置换算法的分级SRAM-Flash硬件接口
		\item 将此设计应用到面向汽车的Cortex-M3微控制器中
	\end{rSubsection}
	
	\begin{rSubsection}{CoNM: 规范微架构核心}{2021年3月 - 2021年6月}{同济大学}{本科毕业设计}
		\item 独立设计一款基于RV32I指令集,带有4级流水线与静态分支预测的50MHzCPU软核
		\item 在PYNQ-Z1 FPGA开发板上成功完成该软核的移植与分析工作
	\end{rSubsection}

	\begin{rSubsection}{数字集成电路课程设计}{2020年6月 - 2020年7月}{同济大学}{课程设计}
		\item 使用Cadence virtuoso设计了一种3结构嵌套的32-bit快速加法器
		\item 完成了该设计的层次、实现、验证与仿真分析
	\end{rSubsection}

	\begin{rSubsection}{A Single-layer Wideband Microwave Absorber with Reactive Screen, A Novel Design of Microwave Absorber for Reduction of Radar Cross Section}{2019年12月 - 2020年5月}{同济大学}{第二作者}
		\item 负责HFSS天线仿真与实验数据分析工作
		\item 以第二作者身份在IEEE AP-S/URSI 2020会议上发表两篇论文
	\end{rSubsection}

\end{rSection}

\newpage

\begin{rSection}{现有项目}
	
	\begin{rSubsection}{ReMap: 一种优化的基于Mitchell方法的近似对数转换电路}{2021年8月 - }{南洋理工大学}{Msc Dissertation Project}
		\item 研究一种Mitchell对数近似转换电路的优化方法
		\item 使用STM40nm工艺综合、仿真并评估该电路
	\end{rSubsection}
	
	\begin{rSubsection}{移植CoNM内核至CHISEL}{2021年10月 - }{自学}{}
		\item 学习CHISEL硬件设计语言并移植CoNM软核
		\item 评估可添加的指令集模块
	\end{rSubsection}

\end{rSection}

\begin{rSection}{所获奖项}
	
	\begin{rSubsection}{同济大学}{2017年第1学期}{暑期实践先进个人}{}
		\item 曾任同济大学暑期实践项目《改革开放后长株潭城市群以交通为中心基建情况的调研》负责人
	\end{rSubsection}
		
	\begin{rSubsection}{同济大学}{2019年第1学期}{优秀学生干部}{}
		\item 曾任同济大学学生会权益保障与生活福利部部长
	\end{rSubsection}

\end{rSection}

\begin{rSection}{专业技能}
	
	\begin{rSubsection}{编程语言}{}{}{}
		\item 擅长使用Verilog与VHDL
		\item 熟练使用C/C++
		\item 具有一定的CHISEL与Python编程能力
	\end{rSubsection}

	\begin{rSubsection}{专业软件}{}{}{}
		\item 擅长使用ModelSim与MATLAB 
		\item 熟练使用Vivado,Synopsys,Cadence,git,MarkDown等
		\item 具备HFSS,ISE,Keil等软件的使用经验
	\end{rSubsection}

	\begin{rSubsection}{语言}{}{母语:中文,熟练:英语}{}
		\item 普通话水平测试90.5/100,二级甲等
		\item IELTS 8.0/9.0,CEFR等级C1
	\end{rSubsection}

\end{rSection}

\begin{rSection}{兴趣爱好}
	\item 文学与语言,音乐,足球
\end{rSection}

\end{document}
